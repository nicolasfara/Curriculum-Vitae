%%%%%%%%%%%%%%%%%%%%%%%%%%%%%%%%%%%%%%%%%
% "ModernCV" CV and Cover Letter
% LaTeX Template
% Version 1.3 (29/10/16)
%
% This template has been downloaded from:
% http://www.LaTeXTemplates.com
%
% Original author:
% Xavier Danaux (xdanaux@gmail.com) with modifications by:
% Vel (vel@latextemplates.com)
%
% License:
% CC BY-NC-SA 3.0 (http://creativecommons.org/licenses/by-nc-sa/3.0/)
%
% Important note:
% This template requires the moderncv.cls and .sty files to be in the same 
% directory as this .tex file. These files provide the resume style and themes 
% used for structuring the document.
%
%%%%%%%%%%%%%%%%%%%%%%%%%%%%%%%%%%%%%%%%%

%----------------------------------------------------------------------------------------
%	PACKAGES AND OTHER DOCUMENT CONFIGURATIONS
%----------------------------------------------------------------------------------------

\documentclass[11pt,a4paper,sans]{moderncv} % Font sizes: 10, 11, or 12; paper sizes: a4paper, letterpaper, a5paper, legalpaper, executivepaper or landscape; font families: sans or roman

\moderncvstyle{casual} % CV theme - options include: 'casual' (default), 'classic', 'oldstyle' and 'banking'
\moderncvcolor{grey} % CV color - options include: 'blue' (default), 'orange', 'green', 'red', 'purple', 'grey' and 'black'
\usepackage[utf8]{inputenc}
\usepackage[scale=0.85]{geometry} % Reduce document margins
%\setlength{\hintscolumnwidth}{3cm} % Uncomment to change the width of the dates column
%\setlength{\makecvtitlenamewidth}{10cm} % For the 'classic' style, uncomment to adjust the width of the space allocated to your name

%----------------------------------------------------------------------------------------
%	NAME AND CONTACT INFORMATION SECTION
%----------------------------------------------------------------------------------------

\name{Nicolas}{Farabegoli} % Your last name

\email{nicolas.farabegoli@gmail.com}
%\email{nicolas.farabegoli2@unibo.it}
\homepage{nicolasfarabegoli.it}
\social[github]{nicolasfara}
%\quote{"A witty and playful quotation" - John Smith}

%----------------------------------------------------------------------------------------

\begin{document}

%----------------------------------------------------------------------------------------
%	CURRICULUM VITAE
%----------------------------------------------------------------------------------------

\makecvtitle

%----------------------------------------------------------------------------------------
%	EDUCATION SECTION
%----------------------------------------------------------------------------------------

\section{Education}

\cventry{2011--2016}{Diploma superiore}{ITT B. Pascal}{Cesena (FC), Italy}{}{Corso di Elettronica}
\cventry{2016--2020}{Laurea Triennale}{Università di Bologna}{Cesena (FC), Italy}{\newline Ingegneria e Scienze Informatiche}{
    \begin{itemize}
        \item Tesi: Implementazione ottimizzata dell'operatore di Dirac su GPGPU
        \item Supervisore: Moreno Marzolla
    \end{itemize}
}

\cventry{2021--Oggi}{Laurea Magistrale}{Università di Bologna}{Cesena (FC), Italy}{\newline Ingegneria e Scienze Informatiche}{In corso}

%----------------------------------------------------------------------------------------
%	WORK EXPERIENCE SECTION
%----------------------------------------------------------------------------------------

\section{Experience}

\cventry{Jun~2015--Aug~2015}{Tirocinio in sistemi embedded}{General System}{}{}{
    \begin{itemize}
        \item Sviluppo di un sistema per in controllo remoto di dispositivi embedded.
        \item Sviluppo e testing di quadri elettrici industriali.
        \item Progettazione di un sistema per il controllo degli accessi basato su tecnologia RFID.
    \end{itemize}
}

\section{Tutorati}

\cventry{Mar~2022--Sep~2022}{Algoritmi e Strutture Dati}{Università di Bologna}{Cesena (FC), Italy}{}{Module A-B}
\cventry{Mar~2021--Sep~2021}{Algoritmi e Strutture Dati}{Università di Bologna}{Cesena (FC), Italy}{}{Module B}

%------------------------------------------------
%   Projects
%------------------------------------------------

\section{Projects}

\cventry{Sep~2020--Mar~2021}{Unibo Motorsport}{Bologna (BO), Italy}{}{}{Membro divisione E-Powertrain}
\cventry{Jan~2016--Feb~2019}{CeSeNA Security Team}{Cesena (FC), Italy}{}{}{
    \begin{itemize}
        \item Membro attivo
        \item Esperienza in reverse engeniiring di binari.
    \end{itemize}
}
\cventry{Sep~2014--Jul~2017}{FabLab Romagna}{Cesena (FC), Italy}{}{}{
    Durante il periodo ho sviluppato diversi prototipi con Arduino e Raspberry, preso parte a progetti di gruppo per eventi e fiere. Sono stato conivolto anche in:
    \begin{itemize}
        \item Organizzazione e migliramento del laboratorio con nuovi strumenti e strumentazioni.
        \item Organizzatore del seminario ``Introduzione all'elettronica e i suoi componenti''
        \item Partecipazione nell'organizzazione dell'evento \textit{Rimini Beach Mini Maker Faire} (2015).
    \end{itemize}
}

%----------------------------------------------------------------------------------------
%   COMPUTER SKILLS SECTION
%----------------------------------------------------------------------------------------

\newpage

\section{Skills}

\cvitem{\textit{Autovalutazione}}{
    \begin{itemize} % TODO(Improve those items)
        \item 1 --- \textit{Piccola esperienza, comprensione di programmi semplici, capacità di apportare alcune modifiche a programmi esistenti}
        \item 2 --- \textit{Esperienza media, capacità di scrivere qualche riga di codice senza alcun aiuto}
        \item 3 --- \textit{Esperienza quotidiana, capacità di utilizzare alcune funzionalità del linguaggio senza alcun aiuto}
        \item 4 --- \textit{Esperienza avanzata, capacità di scrivere programmi idiomatici con risoluzione di problemi non banali}
        \item 5 --- \textit{Esperienza massima, può scrivere programmi avanzati e su larga scala con comportamenti non banali}
    \end{itemize}
}

\cvitem{Linguaggi}{C/C++ (2), Java (4), Kotlin (3), Python (3), TypeScript (3), Scala (4), Ruby (1)}
\cvitem{Frameworks}{Cuda (2), Keras (1), Tensorflow (1)}
\cvitem{Utilities}{Git (4), Jupyter Notebook (2)}
\cvitem{Lingue}{English, Italian}
\cvitem{O.S.}{GNU Linux, Mac, Windows}
\cvitem{Altro}{3D Printer, PCB Designer, SMD assembler, CAD Designer}

%----------------------------------------------------------------------------------------
%   Open Source Contributions
%----------------------------------------------------------------------------------------

\section{Contributo a progetti Open Source}
\cvitem{\href{https://github.com/atedeg/ecscala}{ECScala}}{Sviluppatore e maintainer del repository --- 2021.}
\cvitem{\href{https://github.com/nicolasfara/pfeeder}{pfeeder}}{Sviluppatore e maintainer del repository --- 2021.}
\cvitem{\href{https://gitlab.com/nic0lasfara/dirac-operator}{dirac-operator}}{Sviluppo di un'implementazione ottimizzata dell'operatore di Dirac --- 2019/20.}
\cvitem{\href{https://github.com/nicolasfara/oop17-fag}{fag}}{Sviluppatore e maintainer del repository --- 2017.}


%----------------------------------------------------------------------------------------
%   INTERESTS SECTION
%----------------------------------------------------------------------------------------

\section{Interests}

%\renewcommand{\listitemsymbol}{-~} % Changes the symbol used for lists

\cvlistdoubleitem{MTB}{Walking}
\cvlistdoubleitem{Tennis}{Beach Tennis}
\cvlistdoubleitem{Hacking software e hardware}{Elettronica}
\cvlistdoubleitem{Appassionato del mondo FOSS}{Linux world}

\end{document}
