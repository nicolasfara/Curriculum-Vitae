%%%%%%%%%%%%%%%%%%%%%%%%%%%%%%%%%%%%%%%%%%%%%%%%%%%%%%%%%%%%%%%%%%%%%%%%
%%%%%%%%%%%%%%%%%%%%%% Simple LaTeX CV Template %%%%%%%%%%%%%%%%%%%%%%%%
%%%%%%%%%%%%%%%%%%%%%%%%%%%%%%%%%%%%%%%%%%%%%%%%%%%%%%%%%%%%%%%%%%%%%%%%

%%%%%%%%%%%%%%%%%%%%%%%%%%%%%%%%%%%%%%%%%%%%%%%%%%%%%%%%%%%%%%%%%%%%%%%%
%% NOTE: If you find that it says                                     %%
%%                                                                    %%
%%                           1 of ??                                  %%
%%                                                                    %%
%% at the bottom of your first page, this means that the AUX file     %%
%% was not available when you ran LaTeX on this source. Simply RERUN  %%
%% LaTeX to get the ``??'' replaced with the number of the last page  %%
%% of the document. The AUX file will be generated on the first run   %%
%% of LaTeX and used on the second run to fill in all of the          %%
%% references.                                                        %%
%%%%%%%%%%%%%%%%%%%%%%%%%%%%%%%%%%%%%%%%%%%%%%%%%%%%%%%%%%%%%%%%%%%%%%%%

%%%%%%%%%%%%%%%%%%%%%%%%%%%% Document Setup %%%%%%%%%%%%%%%%%%%%%%%%%%%%

% Don't like 10pt? Try 11pt or 12pt
\documentclass[10pt]{article}
% \immediate\write18{./setup_ruby.sh}
% \immediate\write18{./scholar_scraper.rb}
\newcommand{\sapere}{\texttt{SAPERE}}
\newcommand{\doititle}[2]{\href{http://dx.doi.org/#1}{#2}}

% This is a helpful package that puts math inside length specifications
\usepackage{calc}
\usepackage[utf8]{inputenc} % Consente l'uso caratteri accentati italiani
\usepackage{xcolor}
\usepackage{tabularx}
\usepackage[hyphens]{url}
\usepackage[
backend=biber,
style=ieee,
sorting=ynt,
url=false,
defernumbers=true,
bibencoding=utf8
]{biblatex}
\usepackage[hidelinks]{hyperref}
\usepackage{underscore}
% \usepackage{lmodern}
\usepackage[T1]{fontenc}
\usepackage{textcomp}
\hypersetup{
    colorlinks,
    breaklinks,
    linkcolor=darkblue,
    urlcolor=darkblue,
    anchorcolor=darkblue,
    citecolor=darkblue
}
\addbibresource{bibliography.bib}

% Layout: Puts the section titles on left side of page
\reversemarginpar

%
%         PAPER SIZE, PAGE NUMBER, AND DOCUMENT LAYOUT NOTES:
%
% The next \usepackage line changes the layout for CV style section
% headings as marginal notes. It also sets up the paper size as either
% letter or A4. By default, letter was used. If A4 paper is desired,
% comment out the letterpaper lines and uncomment the a4paper lines.
%
% As you can see, the margin widths and section title widths can be
% easily adjusted.
%
% ALSO: Notice that the includefoot option can be commented OUT in order
% to put the PAGE NUMBER *IN* the bottom margin. This will make the
% effective text area larger.
%
% IF YOU WISH TO REMOVE THE ``of LASTPAGE'' next to each page number,
% see the note about the +LP and -LP lines below. Comment out the +LP
% and uncomment the -LP.
%
% IF YOU WISH TO REMOVE PAGE NUMBERS, be sure that the includefoot line
% is uncommented and ALSO uncomment the \pagestyle{empty} a few lines
% below.
%

\usepackage[paper=a4paper,
           %includefoot, % Uncomment to put page number above margin
           marginparwidth=30.5mm,    % Length of section titles
           marginparsep=1.5mm,       % Space between titles and text
           margin=25mm,              % 25mm margins
           includemp]{geometry}

%% More layout: Get rid of indenting throughout entire document
\setlength{\parindent}{0in}

%% This gives us fun enumeration environments. compactitem will be nice.
\usepackage{paralist}

%% Reference the last page in the page number
%
% NOTE: comment the +LP line and uncomment the -LP line to have page
%       numbers without the ``of ##'' last page reference)
%
% NOTE: uncomment the \pagestyle{empty} line to get rid of all page
%       numbers (make sure includefoot is commented out above)
%
\usepackage{fancyhdr,lastpage}
\pagestyle{fancy}
%\pagestyle{empty}      % Uncomment this to get rid of page numbers
\fancyhf{}\renewcommand{\headrulewidth}{0pt}
\fancyfootoffset{\marginparsep+\marginparwidth}
\newlength{\footpageshift}
\setlength{\footpageshift}
          {0.5\textwidth+0.5\marginparsep+0.5\marginparwidth-2in}
\lfoot{\hspace{\footpageshift}%
       \parbox{4in}{\, \hfill %
                    \arabic{page} of \protect\pageref*{LastPage} % +LP
%                    \arabic{page}                               % -LP
                    \hfill \,}}

% Finally, give us PDF bookmarks
\usepackage{color}
\definecolor{darkblue}{rgb}{0.0,0.0,0.3}


%%%%%%%%%%%%%%%%%%%%%%%% End Document Setup %%%%%%%%%%%%%%%%%%%%%%%%%%%%


%%%%%%%%%%%%%%%%%%%%%%%%%%% Helper Commands %%%%%%%%%%%%%%%%%%%%%%%%%%%%

% The title (name) with a horizontal rule under it
% (optional argument typesets an object right-justified across from name
%  as well)
%
% Usage: \makeheading{name}
%        OR
%        \makeheading[right_object]{name}
%
% Place at top of document. It should be the first thing.
% If ``right_object'' is provided in the square-braced optional
% argument, it will be right justified on the same line as ``name'' at
% the top of the CV. For example:
%
%       \makeheading[\emph{Curriculum vitae}]{Your Name}
%
% will put an emphasized ``Curriculum vitae'' at the top of the document
% as a title. Likewise, a picture could be included:
%
%   \makeheading[\includegraphics[height=1.5in]{my_picutre}]{Your Name}
%
% the picture will be flush right across from the name.
\newcommand{\makeheading}[2][]%
        {\hspace*{-\marginparsep minus \marginparwidth}%
         \begin{minipage}[t]{\textwidth+\marginparwidth+\marginparsep}%
             {\large \bfseries #2 \hfill #1}\\[-0.15\baselineskip]%
                 \rule{\columnwidth}{1pt}%
         \end{minipage}}

% The section headings
%
% Usage: \section{section name}
%
% Follow this section IMMEDIATELY with the first line of the section
% text. Do not put whitespace in between. That is, do this:
%
%       \section{My Information}
%       Here is my information.
%
% and NOT this:
%
%       \section{My Information}
%
%       Here is my information.
%
% Otherwise the top of the section header will not line up with the top
% of the section. Of course, using a single comment character (%) on
% empty lines allows for the function of the first example with the
% readability of the second example.
\renewcommand{\section}[2]%
        {\pagebreak[3]\vspace{1.3\baselineskip}%
         \phantomsection\addcontentsline{toc}{section}{#1}%
         \hspace{0in}%
         \marginpar{
         \raggedright \scshape #1}#2}

% An itemize-style list with lots of space between items
\newenvironment{outerlist}[1][\enskip\textbullet]%
        {\begin{itemize}[#1]}{\end{itemize}%
         \vspace{-.6\baselineskip}}

% An environment IDENTICAL to outerlist that has better pre-list spacing
% when used as the first thing in a \section
\newenvironment{lonelist}[1][\enskip\textbullet]%
        {\vspace{-\baselineskip}\begin{list}{#1}{%
        \setlength{\partopsep}{0pt}%
        \setlength{\topsep}{0pt}}}
        {\end{list}\vspace{-.6\baselineskip}}

% An itemize-style list with little space between items
\newenvironment{innerlist}[1][\enskip\textbullet]%
        {\begin{compactitem}[#1]}{\end{compactitem}}

% An environment IDENTICAL to innerlist that has better pre-list spacing
% when used as the first thing in a \section
\newenvironment{loneinnerlist}[1][\enskip\textbullet]%
        {\vspace{-\baselineskip}\begin{compactitem}[#1]}
        {\end{compactitem}\vspace{-.6\baselineskip}}

% To add some paragraph space between lines.
% This also tells LaTeX to preferably break a page on one of these gaps
% if there is a needed pagebreak nearby.
\newcommand{\blankline}{\quad\pagebreak[3]}
\newcommand{\halfblankline}{\quad\vspace{-0.5\baselineskip}\pagebreak[3]}

% Uses hyperref to link DOI
\newcommand\doilink[1]{\href{http://dx.doi.org/#1}{#1}}
\newcommand\doi[1]{doi:\doilink{#1}}

% For \url{SOME_URL}, links SOME_URL to the url SOME_URL
\providecommand*\url[1]{\href{#1}{#1}}
% Same as above, but pretty-prints SOME_URL in teletype fixed-width font
\renewcommand*\url[1]{\href{#1}{\texttt{#1}}}

% For \email{ADDRESS}, links ADDRESS to the url mailto:ADDRESS
\providecommand*\email[1]{\href{mailto:#1}{#1}}
% Same as above, but pretty-prints ADDRESS in teletype fixed-width font
%\renewcommand*\email[1]{\href{mailto:#1}{\texttt{#1}}}

%\providecommand\BibTeX{{\rm B\kern-.05em{\sc i\kern-.025em b}\kern-.08em
%    T\kern-.1667em\lower.7ex\hbox{E}\kern-.125emX}}
%\providecommand\BibTeX{{\rm B\kern-.05em{\sc i\kern-.025em b}\kern-.08em
%    \TeX}}
\providecommand\BibTeX{{B\kern-.05em{\sc i\kern-.025em b}\kern-.08em
    \TeX}}
\providecommand\Matlab{\textsc{Matlab}}

\newcommand{\shortcourse}[6]{
    \href{#1}{\textit{\textbf{#2}}},
    \href{#3}{#4},
    \href{#5}{#6}}

\newcommand{\course}[9]{
    \shortcourse{#1}{#2}{#3}{#4}{#5}{#6},
    #7,
    #8.
    \textit{#9}
}

\newcommand{\shortunibocourse}[4]{
    \shortcourse{#1}{#2}
    {http://www.unibo.it}{Alma Mater Studiorum---University of Bologna}
    {#3}{#4}}

\newcommand{\unibocourse}[7]{
    \course{#1}{#2}
    {http://www.unibo.it}{Alma Mater Studiorum---University of Bologna}
    {#3}{#4}
    {#5}
    {#6}{#7}
}

\newcommand{\activity}[5]{
    \href{#1}{\textit{\textbf{#2}}},
    {#3},
    {#4},
    {#5}
}

%%%%%%%%%%%%%%%%%%%%%%%% End Helper Commands %%%%%%%%%%%%%%%%%%%%%%%%%%%

%%%%%%%%%%%%%%%%%%%%%%%%% Begin CV Document %%%%%%%%%%%%%%%%%%%%%%%%%%%%

\newcommand{\macrosection}[1]{

\vspace{20pt}
\hrulefill
\begin{center}
\textbf{\MakeUppercase{#1}}
\end{center}
\vspace{-7pt}
\hrulefill
}

\begin{document}
\makeheading{Nicolas Farabegoli}

\section{Contacts}
%
% NOTE: Mind where the & separators and \\ breaks are in the following
%       table.
%
% ALSO: \rcollength is the width of the right column of the table
%       (adjust it to your liking; default is 1.85in).
%
\newlength{\rcollength}\setlength{\rcollength}{2.5in}%
%
\begin{tabular}[t]{@{}p{\textwidth-\rcollength}p{\rcollength}}                          
Via Bagalona, 599          & \textit{Cell:} +39 3402876022\\
47032 Bertinoro (FC)       & \textit{E-mail:} \email{nicolas.farabegoli@unibo.it}\\
Italy                      & \textit{WWW:} \href{https://www.nicolasfarabegoli.it/}{www.nicolasfarabegoli.it}\\
\end{tabular}

\section{Citizenship}
Italy

\section{Current placement}
\textbf{University of Bologna, Department of Computer Science and Engineering} \hfill 
\begin{outerlist}
    \item Research fellow\\\emph{Assegnista di ricerca}
    \item Teaching tutor\\\emph{Tutor didattico}
\end{outerlist}

%\newpage

\macrosection{EDUCATION}

\section{Education}
%
\href{https://disi.unibo.it/it}{\textbf{Department of Computer Science and Engineering, University of Bologna}}, Cesena (FC), Italy
\begin{outerlist}
    \item[] M.S.,
            \href{https://corsi.unibo.it/laurea/IngegneriaScienzeInformatiche}{Computer Science and Engineering}, March 2023
            \begin{innerlist}
                \item \emph{110L/100 - Summa cum Laude}
                \item Thesis Topic: \emph{Design and Implementation of a Portable Framework for Applicationd Decomposition and Deployment in Edge-Cloud Systems}
                \item Supervisors:
                      \href{https://www.unibo.it/sitoweb/mirko.viroli}{Prof. Mirko Viroli}, \href{https://www.unibo.it/sitoweb/danilo.pianini}{Prof. Danilo Pianini}
                \item Area of study: Pervasive Computing
            \end{innerlist}
    \item[] B.S.,
            \href{https://corsi.unibo.it/magistrale/IngegneriaScienzeInformatiche}{Computer Science and Engineering}, March 2020
            \begin{innerlist}
                \item Thesis Topic: \emph{Optimized Implementation of the Dirac Operator on GPGPU}
                \item Supervisor:
                      \href{https://www.unibo.it/sitoweb/moreno.marzolla}{Prof. Moreno Marzolla}
                \item Area of study: High Performance Computing
            \end{innerlist}
\halfblankline
\end{outerlist}

\href{https://www.ispascalcomandini.it/}{ITT Blaise Pascal}, Cesena (FC), Italy
\begin{outerlist}
    \item[] High School Diploma,
            \href{https://www.ispascalcomandini.it/}{Electronics and Automation specialization}, June 2016
            \begin{innerlist}
                \item \emph{90/100}
                \item Final project: \emph{Implementation of a Functions Generator with Raspberry PI}
            \end{innerlist}
\halfblankline
\end{outerlist}

\macrosection{Scientific Pubblications}

\section{Submitted}
\vspace{-1.9em}
\renewcommand{\section}[2]{}
\nocite{*}
\DeclareFieldFormat{labelnumberwidth}{#1}
\printbibliography[omitnumbers=true]{}
\renewcommand{\section}[2]%
        {\pagebreak[3]\vspace{1.3\baselineskip}%
         \phantomsection\addcontentsline{toc}{section}{#1}%
         \hspace{0in}%
         \marginpar{
         \raggedright \scshape #1}#2}

\newpage

\macrosection{Experience}

\section{Tutoring}
\vspace{-1.9em}
\begin{outerlist}
    \item[2022/2023]
        \shortunibocourse
        {https://www.unibo.it/it/didattica/insegnamenti/insegnamento/2022/412181}{Algorithm and Data Structures}
        {https://corsi.unibo.it/laurea/IngegneriaScienzeInformatiche}{Bachelor in Computer Science and Engineering}
    \item[2021/2022]
        \shortunibocourse
        {https://www.unibo.it/it/didattica/insegnamenti/insegnamento/2021/412181}{Algorithm and Data Structures}
        {https://corsi.unibo.it/laurea/IngegneriaScienzeInformatiche}{Bachelor in Computer Science and Engineering}
    \item[2020/2021]
        \shortunibocourse
        {https://www.unibo.it/it/didattica/insegnamenti/insegnamento/2020/412181}{Algorithm and Data Structures}
        {https://corsi.unibo.it/laurea/IngegneriaScienzeInformatiche}{Bachelor in Computer Science and Engineering}
\end{outerlist}

\section{Internships}
\vspace{-1.9em}
\begin{outerlist}
    \item[2019]
        \shortunibocourse
        {https://www.unibo.it/it/didattica/insegnamenti/insegnamento/2019/385080}{HPC Internship focused on NVIDIA GPU programming}
        {https://corsi.unibo.it/laurea/IngegneriaScienzeInformatiche}{Bachelor in Computer Science and Engineering}
    \item[2015]
        \textit{\textbf{High School Internship}} at \href{https://generalsystem.net/}{General System S.r.l.}
\end{outerlist}

\macrosection{Skills}

\section{Programming Languages}
\vspace{-1.9em}
\begin{center}
    \begin{tabular}{p{2cm} p{2cm} p{2cm} p{2cm}}
        Bash & C & C++ & C\# \\
        Kotlin & Java & Javascript & Prolog \\
        Protelis & Python & Rust & Scala \\
        TypeScript
    \end{tabular}
\end{center}

\section{Other Languages}
\vspace{-1.9em}
\begin{center}
    \begin{tabular}{p{2cm} p{2cm} p{2cm} p{2cm}}
        CSS & HTML & JSON & \LaTeX \\
        Markdown & SQL & YAML
    \end{tabular}
\end{center}

\section{Software Tools}
\vspace{-1.9em}
\begin{center}
    \begin{tabular}{p{2cm} p{2cm} p{2cm} p{2cm}}
        Cargo & Git & Gradle & GH Actions \\
        Hugo & Inkscape & Markdown & sbt \\
        SQL & npm & IntelliJ & VS Code \\
    \end{tabular}
\end{center}


\macrosection{Extra Curricular Activities}
\begin{outerlist}
    \item[2020/2021]
        \activity{https://motorsport.unibo.it/}{Unibo Motorsport}
        {E--Powertrain Division Member}
        {Bologna, Italy}
        {
            \begin{innerlist}
                \item Developed BMS boards for bike's battery pack;
                \item Improved \emph{LabView} program for testing battery pack;
                \item Developed the bike's wiring system.
            \end{innerlist}
        }
    \item[2016/2019]
        \activity{https://cesena.github.io/about/}{Ce.Se.N.A. Security Team}
        {Core Member}
        {Cesena, Italy}
        {
            \begin{innerlist}
                \item Learned reverse engineering techniques, mainly focused on binary;
                \item Attended several presentations, as well as seminars focusing on security;
                \item Took part in several \emph{CTF} competitions.
            \end{innerlist}
        }
    \item[2014/2017]
        \activity{https://fablabromagna.org/}{Fablab Romagna}
        {Core Member}
        {Cesena, Italy}
        {
            \begin{innerlist}
                \item In charge of improving the laboratory with new tools and equipment;
                \item Speaker of the seminar ``Basic introduction to electronics and its component";
                \item Involved in the organization of the \emph{Rimini Beach Mini Maker Faire} (2015);
                \item Developed several prototypes with \emph{Arduino} and \emph{Raspberry PI}.
            \end{innerlist}
        }
\end{outerlist}

\newpage

\macrosection{Projects Contributions}

\section{Lead Designer of Software Projects}
Lead Designer and major maintainer of \href{https://github.com/pulvreakt/pulvreakt}{PulvReAKt}, 2022--today
\begin{innerlist}
    \item \emph{PulvReAKt} is a Kotlin framework for the development of distributed applications using the \emph{pulverization approach}.
\end{innerlist}
\halfblankline

Major maintainer of \href{https://github.com/atedeg/mdm}{MDM}, 2022
\begin{innerlist}
    \item \emph{MDM} is a software modelling of the business processes of the \emph{Mambelli} cheese factory.
        The project is written in Scala using a pure functional approach.
\end{innerlist}
\halfblankline

Major maintainer of \href{https://github.com/atedeg/ecscala}{ECScala}, 2021
\begin{innerlist}
    \item \emph{ECScala} is an \emph{ECS} (Entity Component System) framework in Scala.
\end{innerlist}
\halfblankline

Lead designer and major maintainer of \href{https://github.com/nicolasfara/conventional-commits}{conventional-commits}, 2022--today
\begin{innerlist}
    \item \emph{conventional-commits} is a Gradle plugin to enforce the use of the \emph{conventional commits} standard.
\end{innerlist}
\halfblankline

Lead designer and major maintainer of \href{https://github.com/nicolasfara/sbt-conventional-commits}{sbt-conventional-commits}, 2022--today
\begin{innerlist}
    \item \emph{conventional-commits} is am Sbt plugin to enforce the use of the \emph{conventional commits} standard.
\end{innerlist}
\halfblankline

Lead Designer and major maintainer of \href{https://github.com/nicolasfara/pfeeder}{pfeeder}, 2020
\begin{innerlist}
    \item \emph{pfeeder} is a cloud-based software system for the management of a smart pet feeder.
\end{innerlist}

Lead Designer and major maintainer of \href{https://github.com/nicolasfara/oop17-fag}{fag}, 2018
\begin{innerlist}
    \item \emph{fag} is the OOP final course project 2D game.
\end{innerlist}

\section{Contribution to Open Source Projects}
Contributor to \href{https://github.com/gciatto/kt-mpp}{kt-mpp}, 2023--today
\begin{innerlist}
    \item A bundle of Gradle plugins for multi-platform and -project (MPP, henceforth) Kotlin projects.
\end{innerlist}
\halfblankline

Contributor to \href{https://github.com/DanySK/publish-on-central}{publish-on-central}, 2022--today
\begin{innerlist}
    \item \emph{publish-on-central} is a Gradle plugin to publish artifacts on Maven Central.
\end{innerlist}
\halfblankline

Contributor to \href{https://github.com/DanySK/git-sensitive-semantic-versioning-gradle-plugin}{git-sensitive-semantic-versioning-gradle-plugin}, 2022--today
\begin{innerlist}
    \item Gradle plugin to automatically version projects using the \emph{semantic versioning} standard.
\end{innerlist}
\halfblankline

Contributor to \href{https://github.com/DanySK/Template-for-Kotlin-Multiplatform-Projects}{Template-for-Kotlin-Multiplatform-Projects}, 2022--today
\begin{innerlist}
    \item Template for Kotlin Multiplatform Projects.
\end{innerlist}
\halfblankline

Contributor to \href{https://github.com/DanySK/rrmxmx-kt}{rrmxmx-kt}, 2023--today
\begin{innerlist}
    \item \emph{rrmxmx-kt} is a Kotlin library implementing the rrmxmx algorithm: a very fast hashing function that performs very well with low-entropy input data.
\end{innerlist}
\halfblankline

\end{document}