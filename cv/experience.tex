%-------------------------------------------------------------------------------
%	SECTION TITLE
%-------------------------------------------------------------------------------
\langen{\cvsection{Experience}}
\langit{\cvsection{Esperienze}}


%-------------------------------------------------------------------------------
%	CONTENT
%-------------------------------------------------------------------------------
\begin{cventries}
    
    \langen{
        \cventry{Tutor for Alorithm and Data Structures course}
        {University of Bologna}
        {Cesena, Italy}
        {Mar.~2022}
        {
            \begin{cvitems}
                \item {Supporting the teacher helping students in developing exercises during the lesson;}
                \item {Revision of students' end-of-course assignments.}
            \end{cvitems}
        }

        \cventry{Tutor for Alorithm and Data Structures course}
        {University of Bologna}
        {Cesena, Italy}
        {Mar.~2021~--~Sep.2021}
        {
            \begin{cvitems}
                \item {Development of some algorithm exercises;}
                \item {Supporting the teacher helping students in developing exercises during the lesson;}
                \item {Revision of students' end-of-course assignments.}
            \end{cvitems}
        }

        \cventry{High Performance Computing Internship}
        {University of Bologna \& University of Ferrara}
        {Cesena, Italy}
        {Sep.~2019~--~Nov.~2019}
        {
            During the internship activity, expertise was gained in the implementation of efficient programs on NVIDIA Turing GPUs.
            TensorCore units were used to make an optimized version of the Dirac operator.\newline
            Supervisor: Moreno Marzolla
        }

        \cventry{Embedded Systems Internship}
        {General System S.r.l.}
        {Cesena, Italy}
        {Jun.~2015~--~Aug.~2015}
        {
            \begin{cvitems}
                \item {Developed a system for remote control of an embedded board;}
                \item {Developed and tested industrial switchboards;}
                \item {Developed an access control system based on RFID and Arduino technology.}
            \end{cvitems}
        }
    }
    \langit{
        \cventry{Tutor per il corso di Algoritmi e Strutture dati}
        {Università di Bologna}
        {Cesena, Italia}
        {Mar.~2022}
        {
            \begin{cvitems}
                \item {Supporto al docente nell'aiutare gli studenti a sviluppare gli esercizi proposti a lezione;}
                \item {Revisione degli assignment di fine corso degli studenti.}
            \end{cvitems}
        }

        \cventry{Tutor per il corso di Algoritmi e Strutture dati}
        {Università di Bologna}
        {Cesena, Italia}
        {Mar.~2021~--~Sep.2021}
        {
            \begin{cvitems}
                \item {Sviluppo di alcuni esercizi di algoritmi;}
                \item {Supporto al docente nell'aiutare gli studenti a sviluppare gli esercizi proposti a lezione;}
                \item {Revisione degli assignment di fine corso degli studenti.}
            \end{cvitems}
        }

        \cventry{Tirocinio in High Performance Computing}
        {Università di Bologna \& Università di Ferrara}
        {Cesena, Italia}
        {Set.~2019~--~Nov.~2019}
        {
            Durante l'attività di tirocinio è stata acquisita esperienza nell'implementazione di programmi efficienti su GPU NVIDIA Turing.
            Le unità TensorCore sono state utilizzate per realizzare una versione ottimizzata dell'operatore di Dirac.\newline
            Supervisore: Moreno Marzolla
        }

        \cventry{Stage in Sistemi Embedded}
        {General System S.r.l.}
        {Cesena, Italia}
        {Giu.~2015~--~Ago.~2015}
        {
            \begin{cvitems}
                \item {Sviluppo di un sistema per il controllo remoto di una scheda embedded;}
                \item {Realizzazione e collaudo di quadri elettrici;}
                \item {Sviluppo di un sistema di controllo degli accessi basato su tecnologia RFID e Arduino.}
            \end{cvitems}
        }
    }
\end{cventries}