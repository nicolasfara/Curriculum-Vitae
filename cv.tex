%%%%%%%%%%%%%%%%%%%%%%%%%%%%%%%%%%%%%%%%%
% "ModernCV" CV and Cover Letter
% LaTeX Template
% Version 1.3 (29/10/16)
%
% This template has been downloaded from:
% http://www.LaTeXTemplates.com
%
% Original author:
% Xavier Danaux (xdanaux@gmail.com) with modifications by:
% Vel (vel@latextemplates.com)
%
% License:
% CC BY-NC-SA 3.0 (http://creativecommons.org/licenses/by-nc-sa/3.0/)
%
% Important note:
% This template requires the moderncv.cls and .sty files to be in the same 
% directory as this .tex file. These files provide the resume style and themes 
% used for structuring the document.
%
%%%%%%%%%%%%%%%%%%%%%%%%%%%%%%%%%%%%%%%%%

%----------------------------------------------------------------------------------------
%	PACKAGES AND OTHER DOCUMENT CONFIGURATIONS
%----------------------------------------------------------------------------------------

\documentclass[11pt,a4paper,sans]{moderncv} % Font sizes: 10, 11, or 12; paper sizes: a4paper, letterpaper, a5paper, legalpaper, executivepaper or landscape; font families: sans or roman

\moderncvstyle{casual} % CV theme - options include: 'casual' (default), 'classic', 'oldstyle' and 'banking'
\moderncvcolor{grey} % CV color - options include: 'blue' (default), 'orange', 'green', 'red', 'purple', 'grey' and 'black'
\usepackage[utf8]{inputenc}
\usepackage[english,italian]{babel} % Multi-language support
\usepackage[scale=0.85]{geometry} % Reduce document margins
%\setlength{\hintscolumnwidth}{3cm} % Uncomment to change the width of the dates column
%\setlength{\makecvtitlenamewidth}{10cm} % For the 'classic' style, uncomment to adjust the width of the space allocated to your name

\newcommand{\langen}[1]{
  \ifen\selectlanguage{english}#1\fi}
\newcommand{\langit}[1]{
  \ifit\selectlanguage{italian}#1\fi}

%----------------------------------------------------------------------------------------
%	NAME AND CONTACT INFORMATION SECTION
%----------------------------------------------------------------------------------------

\name{Nicolas}{Farabegoli} % Your last name

\email{nicolas.farabegoli@gmail.com}
%\email{nicolas.farabegoli2@unibo.it}
\homepage{nicolasfarabegoli.it}
\social[github]{nicolasfara}
%\quote{"A witty and playful quotation" - John Smith}

%----------------------------------------------------------------------------------------

\begin{document}

%----------------------------------------------------------------------------------------
%	CURRICULUM VITAE
%----------------------------------------------------------------------------------------

\makecvtitle

%----------------------------------------------------------------------------------------
%	EDUCATION SECTION
%----------------------------------------------------------------------------------------

\langen{\section{Education}}
\langit{\section{Educazione}}

\langen{
    \cventry{2020--Ongoing}
    {Master degree}
    {University di Bologna}
    {Cesena (FC), Italy}
    {\newline Ingegneria e Scienze Informatiche}
    {}
}
\langit{
    \cventry{2020--\\In corso}
    {Laurea Magistrale}
    {Università di Bologna}
    {Cesena (FC), Italia}
    {\newline Ingegneria e Scienze Informatiche}
    {}
}

\langen{
    \cventry{2016--2020}
    {Bachelor degree}
    {University di Bologna}
    {Cesena (FC), Italy}
    {\newline Ingegneria e Scienze Informatiche}
    {
        \begin{itemize}
            \item Thesis: \textit{Optimized implementation of Dirac's operator on GPGPU}
            \item Supervisor: Moreno Marzolla
        \end{itemize}
    }
}
\langit{
    \cventry{2016--2020}
    {Laurea Triennale}
    {Università di Bologna}
    {Cesena (FC), Italia}
    {\newline Ingegneria e Scienze Informatiche}
    {
        \begin{itemize}
            \item Tesi: \textit{Implementazione ottimizzata dell'operatore di Dirac su GPGPU}
            \item Supervisore: Moreno Marzolla
        \end{itemize}
    }
}

\langen{
    \cventry{2011--2016}
    {High school diploma}
    {ITT B. Pascal}
    {Cesena (FC), Italy}
    {}
    {Elettronics specializzation}
}
\langit{
    \cventry{2011--2016}
    {Diploma scuola superiore}
    {ITT B. Pascal}
    {Cesena (FC),Italia}
    {}
    {Specializzazione in Elettronica}
}

%----------------------------------------------------------------------------------------
%	WORK EXPERIENCE SECTION
%----------------------------------------------------------------------------------------

\langen{\section{Experience}}
\langit{\section{Esperienze}}

\langen{
    \cventry{Jun~2015--Aug~2015}{Embedded Systems Internship}{General System}{Cesena (FC), Italy}{}{
        \begin{itemize}
            \item Developed a system for remote control of an embedded board;
            \item Developed and tested industrial switchboards;
            \item Developed access control based on RFID technology and Arduino.
        \end{itemize}
    }
}
\langit{
    \cventry{Jun~2015--Aug~2015}{Tirocinio in sistemi embedded}{General System}{Cesena (FC), Italia}{}{
        \begin{itemize}
            \item Sviluppo di un sistema per in controllo remoto di dispositivi embedded;
            \item Sviluppo e testing di quadri elettrici industriali;
            \item Progettazione di un sistema per il controllo degli accessi basato su tecnologia RFID\@.
        \end{itemize}
    }
}


\langen{\section{Tutoring}}
\langit{\section{Tutor didattico}}

\langen{
    \cventry{Mar~2022--Sep~2022}
    {Algoritmi e Strutture Dati}
    {Università di Bologna}
    {Cesena (FC), Italy}
    {}
    {Module A-B}
    \cventry{Mar~2021--Sep~2021}
    {Algoritmi e Strutture Dati}
    {Università di Bologna}
    {Cesena (FC), Italy}
    {}
    {Module B}
}
\langit{
    \cventry{Mar~2022--Set~2022}
    {Algoritmi e Strutture Dati}
    {Università di Bologna}
    {Cesena (FC), Italia}
    {}
    {Module A-B}
    \cventry{Mar~2021--Set~2021}
    {Algoritmi e Strutture Dati}
    {Università di Bologna}
    {Cesena (FC), Italia}
    {}
    {Module B}
}

%------------------------------------------------
%   Projects
%------------------------------------------------

\langen{\section{Projects}}
\langit{\section{Progetti}}

\langen{
    \cventry{Sep~2020--Mar~2021}{Unibo Motorsport}{Bologna (BO), Italy}{}{}{E-Powertrain division member}
    \cventry{Jan~2016--Feb~2019}{CeSeNA Security Team}{Cesena (FC), Italy}{}{}{
        \begin{itemize}
            \item Active member;
            \item Experience in reverse engineering (binary reversing).
        \end{itemize}
    }
    \cventry{Sep~2014--Jul~2017}{FabLab Romagna}{Cesena (FC), Italy}{}{}{
    During the period I developed several prototypes with Arduino and Raspberry, did team projects for events and exhibitions.
    I have also been involved in:
        \begin{itemize}
            \item Organization and improvement of the laboratory with new tools and equipment;
            \item Seminar ``Basic introduction to Electronics and it's components'';
            \item Participation in the organization of the event \textit{Rimini Beach Mini Maker Faire} (2015).
        \end{itemize}
    }
}
\langit{
    \cventry{Set~2020--Mar~2021}{Unibo Motorsport}{Bologna (BO), Italy}{}{}{Membro divisione E-Powertrain}
    \cventry{Gen~2016--Feb~2019}{CeSeNA Security Team}{Cesena (FC), Italy}{}{}{
        \begin{itemize}
            \item Membro attivo;
            \item Esperienza in reverse engineering di binari.
        \end{itemize}
    }
    \cventry{Set~2014--Lug~2017}{FabLab Romagna}{Cesena (FC), Italy}{}{}{
        Durante il periodo ho sviluppato diversi prototipi con Arduino e Raspberry, preso parte a progetti di gruppo per eventi e fiere.
        Sono stato conivolto anche in:
        \begin{itemize}
            \item Organizzazione e migliramento del laboratorio con nuovi strumenti e strumentazioni;
            \item Organizzatore del seminario ``Introduzione all'elettronica e i suoi componenti'';
            \item Partecipazione nell'organizzazione dell'evento \textit{Rimini Beach Mini Maker Faire} (2015).
        \end{itemize}
    }
}

%----------------------------------------------------------------------------------------
%   COMPUTER SKILLS SECTION
%----------------------------------------------------------------------------------------

\newpage

\section{Skills}

\langen{
    \cvitem{\textit{Self evaluation skill}}{
        \begin{itemize} % TODO(Improve those items)
            \item 1 --- \textit{Small exeprience, understand simple programs, make some changes to existing programs}
            \item 2 --- \textit{Medium experience, can write some line of code without any help}
            \item 3 --- \textit{Daily experience, can use some language feature without any help}
            \item 4 --- \textit{Advanced experience, can write idiomatic program with nontrivial throbleshooting}
            \item 5 --- \textit{Master experience, can write advanced and large scale programs with nontrivial behaviuors}
        \end{itemize}
    }

    \cvitem{Languages}{C/C++ (2), Java (4), Kotlin (3), Python (3), TypeScript (3), Scala (4), Ruby (1)}
    \cvitem{Frameworks}{Cuda (2), Keras (1), Tensorflow (1)}
    \cvitem{Utilities}{Git (4), Jupyter Notebook (2)}
    \cvitem{Languages}{English, Italian}
    \cvitem{O.S.}{GNU Linux, Mac, Windows}
    \cvitem{Other}{3D Printer, PCB Designer, SMD assembler, CAD Designer}
}
\langit{
    \cvitem{\textit{Legenda auto-valutazione}}{
        \begin{itemize} % TODO(Improve those items)
            \item 1 --- \textit{Esperienza base, comprensione di semplici programmi, capacità di apportare modifiche}
            \item 2 --- \textit{Esperienza media, capacità di scrivere qualche riga di codice senza alcun aiuto}
            \item 3 --- \textit{Esperienza quotidiana, capacità di utilizzare funzionalità base del linguaggio}
            \item 4 --- \textit{Esperienza avanzata, capacità di scrivere programmi idiomatici e avanzati}
            \item 5 --- \textit{Esperienza massima, capacità di scrivere programmi avanzati e su larga scala}
        \end{itemize}
    }

    \cvitem{Linguaggi}{C/C++ (2), Java (4), Kotlin (3), Python (3), TypeScript (3), Scala (4), Ruby (1)}
    \cvitem{Frameworks}{Cuda (2), Keras (1), Tensorflow (1)}
    \cvitem{Utilities}{Git (4), Jupyter Notebook (2)}
    \cvitem{Lingue}{English, Italian}
    \cvitem{O.S.}{GNU Linux, Mac, Windows}
    \cvitem{Altro}{3D Printer, PCB Designer, SMD assembler, CAD Designer}
}

%----------------------------------------------------------------------------------------
%   Open Source Contributions
%----------------------------------------------------------------------------------------

\langen{\section{Contribution to Open Source projects}}
\langit{\section{Contributi a progetti Open Source}}

\langen{
    \cvitem{\href{https://github.com/atedeg/ecscala}{ECScala}}{Co-developer of the project and repo maintainer --- 2021.}
    \cvitem{\href{https://github.com/nicolasfara/pfeeder}{pfeeder}}{Co-developer of the project and maintainer --- 2021.}
    \cvitem{\href{https://gitlab.com/nic0lasfara/dirac-operator}{dirac-operator}}{Developed an optimized CUDA version of the Dirac operator --- 2019/20.}
    \cvitem{\href{https://github.com/nicolasfara/oop17-fag}{fag}}{Co-developer of the project --- 2017.}
}
\langit{
    \cvitem{\href{https://github.com/atedeg/ecscala}{ECScala}}{Sviluppatore e maintainer del repository --- 2021.}
    \cvitem{\href{https://github.com/nicolasfara/pfeeder}{pfeeder}}{Sviluppatore e maintainer del repository --- 2021.}
    \cvitem{\href{https://gitlab.com/nic0lasfara/dirac-operator}{dirac-operator}}{Sviluppo di un'implementazione ottimizzata dell'operatore di Dirac --- 2019/20.}
    \cvitem{\href{https://github.com/nicolasfara/oop17-fag}{fag}}{Sviluppatore e maintainer del repository --- 2017.}
}



%----------------------------------------------------------------------------------------
%   INTERESTS SECTION
%----------------------------------------------------------------------------------------

\langen{\section{Interests}}
\langit{\section{Interessi}}

%\renewcommand{\listitemsymbol}{-~} % Changes the symbol used for lists
\langen{
    \cvlistdoubleitem{MTB}{Walking}
    \cvlistdoubleitem{Tennis}{Beach Tennis}
    \cvlistdoubleitem{Hacking software and hardware}{Elettronics}
    \cvlistdoubleitem{Open Source enthusiast and contributor}{Linux world}
}
\langit{
    \cvlistdoubleitem{MTB}{Walking}
    \cvlistdoubleitem{Tennis}{Beach Tennis}
    \cvlistdoubleitem{Hacking software e hardware}{Elettronica}
    \cvlistdoubleitem{Appassionato del mondo FOSS}{Linux world}
}

\end{document}
